\documentclass{article}
\usepackage{graphicx} % Required for inserting images

\title{foeda 3}
\author{Marcin Sadowy}
\date{January 2025}

\begin{document}

\maketitle

\section{Equations}
\begin{equation}
    F = -kx
\end{equation}
\begin{equation}
    k = \frac{Gr^4}{NR^3}
\end{equation}
\begin{equation}
    Q = mg
\end{equation}
\begin{equation}
    k = \frac{mg}{x}
\end{equation}
\begin{equation}
    T = 2 \pi \sqrt{\frac{m}{k}}
\end{equation}
\begin{equation}
    T^2 = \frac{4 \pi^2}{k}m
\end{equation}
\begin{equation}
    \frac{1}{k_{eff}} = \sum_{i=1}^N \frac{1}{k_i}
\end{equation}

\section{Uncertainty equations}
\begin{equation}
    u_A(x) = \sqrt{\frac{1}{n(n-1}\sum_{i=1}^N(x_i-x)^2}
\end{equation}
\begin{equation}
    u_B(x) = \sqrt{\frac{(\Delta_p x)^2}{3} + \frac{(\Delta_e x)^2}{3}}
\end{equation}
\begin{equation}
    u(x) = \sqrt{u_A^2(x) + u_B^2(x)}
\end{equation}

\section{Measurement Tables}
In table 1 we collected the measured masses of 11 weights, we also calculated their uncertainties from Eqs. (8),(9) and (10). The type A uncertainty differs depending on the measurement, but the type B uncertainty is constant and equal to: \\
\begin{equation}
    u_B(m) = \sqrt{\frac{(0.01 g)^2}{3} + \frac{(0.02 g)^2}{3}} = 0.012910 \mathrm{g} \\
\end{equation}
Table 2 shows the results of the deformation measurement for the first spring. Once again we calculated the uncertainties of the deformation using Eqs. (8),(9) and (10). The type B uncertainty is once again constant and equal to: \\
\begin{equation}
    u_B(x) = \sqrt{\frac{(0.01 g)^2}{3} + \frac{(0.02 g)^2}{3}} = 0.012910 \mathrm{g} \\
\end{equation}
This quantity stays the same in the measurement of the deformation of the second spring and of the springs connected in series, so we will not repeat it. The notation \(m_i - m_j\) indicates the sum of the masses from index i to index j \\ 

Figure 1 shows the dependence of the deformation of the first spring on the mass. \\
Figure 1 \\
We can determine the formula for the stiffness coefficient \(k\) from Eq. (4):
\begin{equation}
    x = \frac{g}{k}m
\end{equation}
Therefore, the stiffness coefficient can be obtained by calculating the slope coefficient of the x(m) function, which is \(a_1 = 0.401 \mathrm{\frac{m}{kg}}\). \\
We can then solve for k by dividing the slope coefficient by the acceleration due to Earth's gravity. \\
\begin{equation}
    k = \frac{g}{a}
\end{equation}
So, the stiffness coefficient of the first spring is: 
\begin{equation}
    k_1 = \frac{9.81}{0.401} = 24.525 \mathrm{\frac{kg}{s^2}}
\end{equation}


Table 3 shows the results of the deformation measurement for the second spring. \\ 

Figure 2 shows the dependence of the deformation of the second spring on the mass. \\
Figure 2 \\
We can see that the slope coefficient of the function is equal to \(a_2 = 0.394 \mathrm{\frac{m}{kg}}\) \\ 
Using Eq. (14) we can calculate the stiffness coefficient of the second spring: \\
\begin{equation}
    k_2 = \frac{9.81}{0.394} = 24.898 \mathrm{\frac{kg}{s^2}}
\end{equation}


Table 4 shows the results of the deformation measurement for the system of two springs connected in series. \\
Figure 3 shows the dependence of the deformation of the system of springs on the mass. \\ 
The slope coefficient is equal to: \(a_3 = 0.739 \mathrm{\frac{m}{kg}}\). \\
Once again we derive the stiffness coefficient of the system of springs using Eq. (14): \\ 
\begin{equation}
    k_3 = \frac{9.81}{0.739}  = 13.275 \mathrm{\frac{kg}{s^2}}
\end{equation}
The result is close to the one that can be derived using Eq (7). If we solve for \(k_{eff}\) we obtain that:
\begin{equation}
    k_{eff} = \frac{k_1 \cdot k_2}{k_1 + k_2}
\end{equation}
So, in this case \(k_3 = \frac{k_1 \cdot k_2}{k_1 + k_2} = \frac{24.525 \times 24.898}{24.525+24.898} = 12.355 \mathrm{\frac{kg}{s^2}}\) \\
The result is close to the one obtained from the slope coefficient.

Then We calculated the stiffness coefficient of the second spring with the dynamic method. 
Table 5 shows the result of the measurement of 30 oscillations for different masses. The table also includes the period, type A uncertainty as well as the standard combined uncertainty of each time measurement, which are obtained from Eqs. (8) and (10). The type B uncertainty is not in the table as it is constant and can be calculated by using Eq. (9): \\
\begin{equation}
    u_B(t) = \sqrt{\frac{(0.01 s)^2}{3} + \frac{(0.5 s)^2}{3}} = 0.28873 \mathrm{s}
\end{equation}
Figure 4 shows the dependence of the square of the period on the mass. \\
Figure 4. \\ 
From Eq. (6) we can observe that the slope coefficient of the dependence is equal to \(a = \frac{4 \pi ^2}{k}\). By solving for k we arrive at the equation:
\begin{equation}
    k = \frac{4 \pi ^2}{a}
\end{equation}
We can observe on the figure that the slope coefficient of the function is: \(a_4 = 1.503 \mathrm{\frac{s^2}{kg}}\) \\ 
We then calculate the spring coefficient k of the second spring using Eq. (17): \\
\begin{equation}
    k_4 = \frac{4 \pi ^2}{1.503} = 26.266 \mathrm{\frac{kg}{s^2}}
\end{equation}
We can observe that the obtained coefficient is close in value to \(k_1\), which is equal to \(24.898 \mathrm{\frac{kg}{s^2}}\). The difference is affected mainly by the time uncertainty of each measurement. 
\section{Uncertainties of the obtained stiffness coefficient}
In the static method, where the stiffness coefficient is expressed by eq. (14), we can calculate u(k) using propagation of uncertainty. \\
We define the sensitivity coefficients:
\begin{equation}
    c_g = \frac{1}{a}
\end{equation}
\begin{equation}
    c_a = -\frac{g}{a^2}
\end{equation}
Table 6 shows the values of the sensitivity coefficients, as well as the uncertainties of the used quantities for the deformation measurement of the first spring. For the calculations we use the uncertainty of g \(u(g) = 0.01 \mathrm{\frac{m}{s^2}}\). The uncertainty of the slope coefficients was obtained using the LINEST function in Microsoft Excel
\begin{table}[h!]
\centering
\caption{Table 6: Summary results of measurement uncertainty values for the deformation measurement of the first spring}
\begin{tabular}{|c|c|c|c|}
\hline
\textbf{\(x_j\)} & \textbf{\(c_{x_j}\)} & \textbf{\(u(x_j)\)} \\ \hline
\(g \mathrm{\frac{m}{s^2}}\) & 2.4938 \(\mathrm{\frac{kg}{m}}\) & 0.01 \(\mathrm{\frac{m}{s^2}}\) \\ \hline
\(a \mathrm{\frac{m}{kg}}\) & -61.007 \(\mathrm{\frac{kg^2}{s^2 \cdot m}}\) & 0.004977 \(\mathrm{\frac{m}{kg}}\) \\ \hline
\end{tabular}
\end{table}
We then repeat the same calculations for the measurements of the deformations of the second spring, as well as the series of two springs connected in series. \\ 
Table 7 shows the values of the sensitivity coefficients and the uncertainties used to determine the uncertainty of the stiffness coefficient for the measurement of the deformation of the second spring. \\
\begin{table}[h!]
\centering
\caption{Table 7: Summary results of measurement uncertainty values for the deformation measurement of the second spring}
\begin{tabular}{|c|c|c|c|}
\hline
\textbf{\(x_j\)} & \textbf{\(c_{x_j}\)} & \textbf{\(u(x_j)\)} \\ \hline
\(g \mathrm{\frac{m}{s^2}}\) & 2.5381 \(\mathrm{\frac{kg}{m}}\) & 0.01 \(\mathrm{\frac{m}{s^2}}\) \\ \hline
\(a \mathrm{\frac{m}{kg}}\) & -63.194 \(\mathrm{\frac{kg^2}{s^2 \cdot m}}\) & 0.00414 \(\mathrm{\frac{m}{kg}}\) \\ \hline
\end{tabular}
\end{table}\\
Table 8 shows the values of the sensitivity coefficients and the uncertainties used to determine the uncertainty of the stiffness coefficient for the measurement of the deformation of the system of two springs connected in series.
\begin{table}[h!]
\centering
\caption{Table 8: Summary results of measurement uncertainty values for the deformation measurement of the system of two springs connected in series}
\begin{tabular}{|c|c|c|c|}
\hline
\textbf{\(x_j\)} & \textbf{\(c_{x_j}\)} & \textbf{\(u(x_j)\)} \\ \hline
\(g \mathrm{\frac{m}{s^2}}\) & 1.3532 \(\mathrm{\frac{kg}{m}}\) & 0.01 \(\mathrm{\frac{m}{s^2}}\) \\ \hline
\(a \mathrm{\frac{m}{kg}}\) & -17.963 \(\mathrm{\frac{kg^2}{s^2 \cdot m}}\) & 0.011677 \(\mathrm{\frac{m}{kg}}\) \\ \hline
\end{tabular}
\end{table}\\
The formula for the uncertainty of the stiffness coefficient is:
\begin{equation}
    u(k) = \sqrt{c_g^2u^2(g) + c_a^2u^2(a)}
\end{equation}
Now we can determine the combined uncertainties of the stiffness coefficients of the first spring, the second spring, and the system of two springs connected in series. \\
\begin{equation}
    u_c(k_1) = \sqrt{2.4938^2 \times 0.01^2 + (-61.007)^2 \times 0.004977^2} = 0.30465 \mathrm{\frac{kg}{s^2}}
\end{equation}
\begin{equation}
    u_c(k_2) = \sqrt{2.5381^2 \times 0.01^2 + (-63.194)^2 \times 0.00414^2} = 0.26285 \mathrm{\frac{kg}{s^2}}
\end{equation}
\begin{equation}
    u_c(k_3) = \sqrt{1.3532^2 \times 0.01^2 + (-17.963)^2 \times 0.011677^2} = 0.21019 \mathrm{\frac{kg}{s^2}}
\end{equation}
When it comes to the dynamic method, we also use propagation of uncertainty, but for Eq. (20) \\
The sensitivity coefficient is:
\begin{equation}
    c_a = -\frac{4 \pi ^2 }{a^2}
\end{equation}
We calculate the uncertainty of the spring coefficient of the dynamic method using Eq. (24).
\begin{equation}
    u_c(k_4) = \sqrt{(-\frac{4 \pi ^2}{1.503^2})^2 \times 0.004159^2} = 0.072683 \mathrm{\frac{kg}{s^2}}
\end{equation}
We can now calculate the expanded uncertainties of the stiffness coefficients for each measurement. We will be using the formula:
\begin{equation}
    U(k) = k \cdot u_c(k) = 2 \cdot u_c(k)
\end{equation}
We calculate:
\begin{equation}
    U(k_1) = 2 \times 0.30465 = 0.60930 \mathrm{\frac{kg}{s^2}}
\end{equation}
\begin{equation}
    U(k_2) = 2 \times 0.26285 = 0.52570 \mathrm{\frac{kg}{s^2}}
\end{equation}
\begin{equation}
    U(k_3) = 2 \times 0.21019 = 0.42038 \mathrm{\frac{kg}{s^2}}
\end{equation}
\begin{equation}
    U(k_4) = 2 \times 0.072683 = 0.145366 \mathrm{\frac{kg}{s^2}}
\end{equation}
\section{Results}
Finally we arrive at the obtained stiffness coefficients and their uncertainties:
\begin{equation}
    k_1 = 25(0.30) \mathrm{\frac{kg}{s^2}}, k_1 = (25 \pm 0.61) \mathrm{\frac{kg}{s^2}}
\end{equation}
\begin{equation}
    k_2 = 25(0.26) \mathrm{\frac{kg}{s^2}}, k_1 = (25 \pm 0.53) \mathrm{\frac{kg}{s^2}}
\end{equation}
\begin{equation}
    k_3 = 12(0.21) \mathrm{\frac{kg}{s^2}}, k_1 = (12 \pm 0.42) \mathrm{\frac{kg}{s^2}}
\end{equation}
\begin{equation}
    k_4 = 26(0.072) \mathrm{\frac{kg}{s^2}}, k_1 = (26 \pm 0.15) \mathrm{\frac{kg}{s^2}}
\end{equation}
The stiffness coefficient obtained from the dynamic method has substantially lower uncertainties than the stiffness coefficients obtained from the static method, which shows its superior precision.
\end{document}